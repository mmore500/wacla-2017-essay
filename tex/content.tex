Coming of age as a musician and scientist, the mentorship I received --- a laboratory internship, playing alongside older oboists --- profoundly influenced me.
As I make a foothold in my own career, I strive to follow in the footsteps of my role models and bring new members into the communities that I care about.

Unfortunately, systemic barriers to true inclusivity exist in these communities --- especially STEM.
Lived experience has deepened my perspective on these barriers.
I came out as gay the summer before last.
On occasion I am forced to confront the questions ``Do I belong here?'' or even ``Am I safe here?''
When I heard my father weep on the phone after I came out him, I wondered if I belong in our faith.
Holding hands with my boyfriend at the mall, I wonder what assumptions passersby project upon us.
I want to participate as a whole person in the communities I care about.
Although the barriers to belonging I have encountered pale next to those faced by others, I hope to leverage these molehills to reach out over mountains.

I came to understand the importance of the initiative to reach out in a first period credit-recovery Geometry class at Oakland High School.
The first few days serving as a classroom assistant, I'd mill about and occasionally ask students if they had any questions.
My offers got no takers.
My third day, I overheard two students complaining about their chemistry class.
I hopped in, ``Oh, that's nothing...''
After swapping stories for a few minutes, one of the girls turned to me and asked, ``Can you explain this triangle problem to me?''
Although I had never consciously decided to act aloof, I realized that to serve these students I would have to choose \textit{not} to.
Enacting inclusivity boils down this: recognizing decisions that might otherwise slip by unexamined and having the courage to act according to your own compass.

An inclusive community is more vibrant.
In STEM, lived experiences are often incorrectly dismissed as moot.
Although STEM deals in empirical facts, dialogue in STEM is nevertheless grounded in human experience.
My oboe instructor exemplifies the rich intersection of pedagogy and identity.
With a sarcastic wit magnified through lens of his identity as a gay man, Professor Williams teaches with special directness and indelibility.
I will always remember his chiding, after I had performed an etude without adequate musical discretion, against playing ``marching band-issimo.''
His rhetoric is not merely for entertainment; he wields it as a teaching tool.
Professor Williams' remark led me to consider why I had played without musical intent.
I have not played ``marching band-issimo'' since.
I know that I have unique perspective to offer in my intended field.
I know that the same is true for many other students.
At the end of the day, we scientists will be better off authentically engaging with each other in the communities we care about.
To get there, we must take to heart that inclusivity hinges on action, not aspiration.
